% model_generative-process.tex
%
% Copyright (C) 2019 Önder Kartal
%
% Methylation inheritance model - generative process

\documentclass[preview]{standalone}
\usepackage{amsmath}

\begin{document}
\begin{enumerate}
\item Draw the initial methylation probability (i.e. frequency of
  methylated alleles) of the founder population:
  \begin{displaymath}
    \mu_0 \sim \mathrm{Beta}(\pi; \gamma)
  \end{displaymath}
\item For each line \( k=1 \ldots L \), draw the number of methylated
  alleles for the first generation:
  \begin{displaymath}
     Z_{k,1} \sim \mathrm{Binomial}(2N_{k,1}; \mu_{0})
  \end{displaymath}
\item For each lineage and each generation \(t=1 \ldots T\), where
  \(T\) is the number of transitions, draw the number of methylated
  alleles in the subsequent generation:
  \begin{displaymath}
    Z_{k,t+1} | Z_{k,t} \sim \mathrm{Binomial}(2N_{k,t+1}; \mu_{k,t})
  \end{displaymath}
\item
\item The read count for 5mC (\(m_i\)) follows a binomial distribution
  with the number of trials being the total read count (\(n_i\)) and
  the probability of success being \(p_i = M_i / 2\), where \(M_i\) is
  the number of 5mC alleles in the diploid.
  \begin{displaymath}
    m_i \sim \mathrm{Binomial}(n_i, \frac{M_i}{2})
  \end{displaymath}
\item The number of 5mC alleles in the diploid (\(M_i\)) follows a
  binomial distribution with two trials and the probability of success
  being equal to the allele frequency of 5mC among the gametes,
  \(\pi\). (what about selfing since we have a different freq from one
  mother to the other, this \(\pi\) is specific to the mother of plant
  \(i\)?)
  \begin{displaymath}
     M_i \sim \mathrm{Multinomial}(2, \pi)
  \end{displaymath}
\end{enumerate}
\begin{align}
  Z_{t+1}|Z_t &\sim \mathrm{Binomial}(2N, g(x_t))
  \\ \mathrm{logit}(\pi_{it}) &\sim \mathrm{Normal}(\Theta X^t\beta)
\end{align}
\end{document}
